% Versione branch 8uno
\documentclass[a4paper]{article}

% draft/final onecolumn/twocolumn oneside/twoside openany/openright
% Circa 80 lettere per pagina con draft al posto di final fa il pdf senza foto


\usepackage[english, italian]{babel}
% per la sillabazione corretta e per biblatex
% \usepackage[english]english serve se c'e' bisogno di una parte da scrivere in inglese

% https://tex.stackexchange.com/questions/82993/how-to-change-the-name-of-document-elements-like-figure-contents-bibliogr

\usepackage[sc]{mathpazo}

\usepackage[T1]{fontenc}
\usepackage[utf8]{inputenc}
% \usepackage{geometry}
% \geometry{verbose,tmargin=2.5cm,bmargin=2.5cm,lmargin=2.5cm,rmargin=5cm}
% \usepackage[nochapters,pdfspacing,dottedtoc]{classicthesis}
% \setcounter{secnumdepth}{4} % numerini alle sezioni dentro al testo
% \setcounter{tocdepth}{4} % nell'indice
\usepackage{url}
\usepackage{comment}

% hyperref serve per i link
% Tutti le opzioni che ci sono adesso non sappiamo se servono: le disattivo per il momento
% 
\usepackage[]{hyperref} % caricare questo prima di floats per evitare casini
% unicode=true, pdfusetitle, backref=false, pdfborder={0 0 1},
% bookmarks=true, bookmarksnumbered=true, bookmarksopen=true, bookmarksopenlevel=2, colorlinks=false,
% breaklinks=false

%% xcolor serve per colorare il testo
\usepackage[dvipsnames]{xcolor}
%% Questa parte sotto controlla i colori dei riferimenti cliccabili
\hypersetup{colorlinks,
  linkcolor={red!50!black},
  citecolor={red},
  urlcolor={blue!80!black},
  pdfstartview={XYZ null null 1}
}
% \usepackage{textcomp}
% Sperimentale per le liste; non attivare
%% \usepackage{enumitem}
%% \setlist{nosep} %

%%\usepackage{longtable} % per avere tabelle spezzate tra le pagine
\usepackage{tabularx} % per le tabelle con molto testo
\usepackage{lscape} % per ruotare qualche pagina in landscape
\usepackage{booktabs} % per fare tabelle semplici come nei libri (top mid e bottomrule)

\usepackage{setspace} % permette di cambiare la spaziatura quando necessario
\doublespacing % per avere spazio per scrivere a penna
%\onehalfspacing  % per l'interlinea finale
% meglio questi due comandi sopra che linespread.
% Cosi sistema interlinea 1 nelle tabelle automaticamente
% \linespread{1.3} % Di default è \linespread{1} che corrisponde
% all'interlinea uno; 1.3 sta per interlinea 1 e mezzo

% %% \usepackage{multirow} % tabelle multiriga e multicolonna

% %% \usepackage{threeparttable}
% % vedi http://www.tug.org/pracjourn/2007-2/asknelly/

% %% \usepackage{lineno}
% %% \linenumbers
% % Queste due righe sopra aggiungono la numerazione delle righe
% % scommentare nella versione finale

% % Queste righe sotto si occupano della bibliografia
% % \usepackage[babel]{csquotes}


% \usepackage[backend=biber, sorting=nyt, style=authoryear,
%             autolang=hyphen, natbib=true, backref]{biblatex}
% %% style=numeric

% % nty Sort by name, title, year.
% % nyt Sort by name, year, title.
% % nyvt Sort by name, year, volume, title.
% % anyt Sort by alphabetic label, name, year, title.
% % anyvt Sort by alphabetic label, name, year, volume, title.
% % ynt Sort by year, name, title.
% % ydnt Sort by year (descending), name, title.
% % none Do not sort at all. All entries are processed in citation order.
% % Debug Sort by entry key. This is intended for debugging only.

% %% Qui sotto ci va il nome del file della bibliografia
% \bibliography{bibliomone.bib}
% % non ho capito a che serve ma dovrebbe essere utile a citare R in bib
% % \newcommand{\code}[1]{\texttt{\smaller #1}}

% \DeclareBibliographyCategory{cartaceo}
% \DeclareBibliographyCategory{web}

% \defbibheading{cartaceo}{\subsection*{Bibliografia}}
% \defbibheading{web}{\subsection*{Siti Web consultati}}

%% siunitx serve per le unità di misura
\usepackage{siunitx}
\sisetup{detect-all = true, detect-family = true, output-decimal-marker = {,}}
% http://tex.stackexchange.com/questions/66713/can-i-make-siunitx-commands-use-serif-fonts-like-the-rest-of-the-math-in-beamer

%% questa roba sotto serve per non perdere il filo sulle copie stampate
%% commentare opportunamente nella versione finale
\usepackage[short, hhmmss]{datetime}
%\usepackage{fancyhdr}
\usepackage{placeins}
\usepackage{graphicx} %per le figure
%\pagestyle{fancy}
% \setlength{\headheight}{14pt} % il compilatore si lamenta che c'e' poco spazio 1 ott 2016 Siena
\usepackage[]{subfig} % per le figure affiancate

\usepackage{float}
% per forzare il posizionamento delle figure

%% \usepackage[normalem]{ulem}
% serve per barrare del testo in fase di correzione
% se non si usa [normalem], \emph sottolinea invece di fare il suo dovere

%% Pezzo per scrivere note a margine
\usepackage{xargs} % Use more than one optional parameter in a new commands
\usepackage[colorinlistoftodos, prependcaption, italian, textsize=scriptsize]{todonotes}
\newcommandx{\dubbio}[2][1=]{\todo[linecolor=red,backgroundcolor=red!25,bordercolor=red,#1]{#2}}
\newcommandx{\cambia}[2][1=]{\todo[linecolor=blue,backgroundcolor=blue!25,bordercolor=blue,#1]{#2}}
\newcommandx{\info}[2][1=]{\todo[linecolor=OliveGreen,backgroundcolor=OliveGreen!25,bordercolor=OliveGreen,#1]{#2}}
\newcommandx{\migliora}[2][1=]{\todo[linecolor=Plum,backgroundcolor=Plum!25,bordercolor=Plum,#1]{#2}}
\newcommandx{\fatto}[2][1=]{\todo[linecolor=gray, backgroundcolor=black!75]{#2}}
\newcommandx{\nonvisibile}[2][1=]{\todo[disable,#1]{#2}}
% \presetkeys{todonotes}{fancyline}{}
% Vediamo se con questo si riduce lo spazio tra la caption.
\usepackage[font=small, textfont=it, labelfont=bf, format=plain]{caption}


% \usepackage[printwatermark]{xwatermark}
% \newwatermark[allpages,color=grey!50,angle=90,scale=1,xpos=0,ypos=0]{Tesi
% di Simone Massenzio}

%\usepackage[raggedright]{titlesec}
% Serve per evitare la sillabazione nei titoli

%\newcommand{\sectionbreak}{\clearpage}
%% serve per mettere ogni section a pagina nuova

%\renewcommand{\chaptermark}[1]{\markboth{\thechapter.\space#1}{}}
%\renewcommand{\headrulewidth}{0.4pt}
%\renewcommand{\footrulewidth}{0.2pt}
% \cfoot{Pag. \thepage\, Compilata il \today\ alle \currenttime}
% \lfoot{}
% \rfoot{}
% E Even page
% O Odd page
% L Left field
% C Center field
% R Right field
% H Header
% F Footer
% \fancyhead{} % clear all header fields
\usepackage{enumitem}
% per fare l'itemize con il tondo intorno ai numeri
% e altri controlli sulle liste

%\usepackage[font=it, labelfont=bf, format=plain, skip=40pt]{caption}
% The optional argument of \setcapwidth is not supported and will be ignored if used in con-
% junction with the caption package. Furthermore the KOMA-Script options tablecaption-
% 30above & tablecaptionbelow and the commands \captionabove & \captionbelow
% are stronger than the position= setting offered by the caption package.
% \DeclareCaptionFont{mysize}{\fontsize{10}{9.6}\selectfont}
% \captionsetup{font=mysize}
% \usepackage[swapnames]{frontespizio} %per il frontespizio

\renewcommand{\theequation}{Eq. \arabic{equation}}
\usepackage{environ}
\NewEnviron{modello}{%
  \begin{equation}
    \BODY
  \end{equation}
}

\usepackage{amsmath}
%\numberwithin{equation}{chapter} % numera le equazioni col numero di capitolo


\DeclareSIUnit\atm{atm}
\DeclareSIUnit\torr{torr}
\usepackage{bm}


\begin{document}
%
\section{Densità apparente}


\subsection{Metodo \emph{Core}}

La tabella \ref{tab:summaryCore} riporta i valori medi per la densità
apparente ottenuta col cilindro in campo. La figura
\ref{fig:boxplotCore} riporta gli stessi dati in forma grafica. I
valori, la cui media generale \`e pari a
\SI{1.37}{\gram\per\cubic\centi\metre}, sono simili a quelli che pi\`u
frequentemente si riscontrano nel suolo.


% latex table generated in R 3.4.2 by xtable 1.8-2 package
% Sun Nov  5 14:09:36 2017
\begin{table}[hb]
\centering
\caption{Valori medi della densità apparente (in \SI{}{\gram\per\cubic\centi\metre}),
suddivisi per \emph{Anno}, \emph{Conduzione} e \emph{Lavorazione} } 
\label{tab:summaryCore}
\begin{tabular}{lllccc}
  \hline
Anno & Conduzione & Lavorazione & Media & Dev. std & n \\ 
  \hline
2015 & Convenzionale & Arato & 1.35 & 0.14 &   5 \\ 
    &   & Frangizollato & 1.34 & 0.07 &   5 \\ 
    &   & Rippato & 1.38 & 0.10 &   5 \\ 
\\
    & Organico & Arato & 1.38 & 0.07 &   5 \\ 
    &   & Frangizollato & 1.42 & 0.13 &   5 \\ 
    &   & Rippato & 1.43 & 0.04 &   5 \\ 
\\
\\
  2016 & Convenzionale & Arato & 1.38 & 0.16 &   6 \\ 
    &   & Frangizollato & 1.35 & 0.15 &   6 \\ 
    &   & Rippato & 1.26 & 0.09 &   5 \\ 
\\
    & Organico & Arato & 1.37 & 0.06 &   6 \\ 
    &   & Frangizollato & 1.35 & 0.07 &   6 \\ 
    &   & Rippato & 1.40 & 0.15 &   6 \\ 
\hline
\end{tabular}
\end{table}\FloatBarrier



\begin{figure}[hb]
  \centering
  % <<label=figboh2, fig=TRUE, echo=FALSE, width=5.4, height=5.4 >>=
  % <<ResiduiCampo>>
  % @
  \includegraphics[width=\textwidth]{Tesi_GIT-figboh2.pdf}

  \caption[Boxplot della densità apparente: metodo
  \emph{Core}]{Boxplot della densità apparente misurata con il metodo
    \emph{Core} relativi alla conduzione (\emph{Convenzionale} e
    \emph{Biologico}) e per le varie lavorazioni (\emph{Arato,
      Frangizollato} e \emph{Rippato}). La riga rossa indica la media
    generale che non si discosta dai valori che frequentemente si
    riscontrano nel terreno }
  \label{fig:boxplotCore}
\end{figure}
\FloatBarrier


%% Il modello utilizzato, (\ref{mod:MetodoClod}, pag.
%% \pageref{mod:MetodoClod}). \`E stato costruito un modello additivo in
%% quanto non sono state trovate interazioni statisticamente
%% significative tra l'anno, la \emph{Conduzione} e la \emph{Lavorazione}

%% \migliora[inline]{riscrivere}

La tabella \ref{tab:anova_del_modello} riporta l'ANOVA del modello
statistico e indica la non significatività dei trattamenti presi in
esame.

\todo[inline]{scrivere modello}

% latex table generated in R 3.4.2 by xtable 1.8-2 package
% Sun Nov  5 14:09:36 2017
\begin{table}[ht]
\centering
\caption{Tabella ANOVA per i valori di densità rilevati col metodo \emph{Core}} 
\label{tab:anova_del_modello}
\begin{tabular}{lrrrrr}
  \hline
 & Df & Sum Sq & Mean Sq & F value & Pr($>$F) \\ 
  \hline
Anno & 1 & 0.02 & 0.02 & 1.41 & 0.2390 \\ 
  Conduzione & 1 & 0.04 & 0.04 & 3.29 & 0.0745 \\ 
  Lavorazione & 2 & 0.00 & 0.00 & 0.03 & 0.9728 \\ 
  Totale & 60 & 0.71 & 0.01 &  &  \\ 
   \hline
\end{tabular}
\end{table}




\subsection{Metodo \emph{Clod}}


Il metodo \emph{Clod} ha permesso, grazie alla sua speditività, di
analizzare tutte le 108 combinazioni dei fattori dell'esperimento: per
ogni combinazione sono stati analizzati tre campioni. La media dei tre
campioni è stata usata per l'analisi dei dati. I valori della densità
apparente calcolati sono mediamente pi\`u elevati (media; 1.89)
rispetto a quelli ottenuti utilizzando il metodo \emph{Core}.
\todo[inline]{manca valore della media}.

La differenza tra i due metodi di misura è ascrivibile al fatto che
col cilindro si prendono in considerazione pori di grandi dimensioni
(fratture) che vengono esclusi automaticamente dalla piccola
dimensione degli aggregati (2-3 cm diametro).


In tabella \ref{tab:summaryClod} \`e riportata l'analisi della
varianza per la densità apparente degli aggregati che mostra alcune
differenze significative (tukey test <0.05) ma che non sembrano
ascrivibili ai trattamenti considerati nell'esperimento.

Inoltre le differenze osservate ricadono in una variazione della
densità alla seconda cifra decimale, ovvero differenze
tecnologicamente e praticamente di poco momento. Esse sono la
conseguenza dell'elevato numero di campioni analizzati (108).



\begin{figure}[ht]
  \centering
\includegraphics{Tesi_GIT-figmah}
\caption[Boxplot dei valori di densità apparente: metodo \emph{Clod}
]{Boxplot dei valori di densità apparente dei campioni misurati col
  metodo di analisi \emph{Clod}; gli appezzamenti condotti
  convenzionalmente (a sinistra) hanno valori più alti degli
  appezzamenti condotti a Biologico (a destra)}
  \label{fig:boxplotClod}

\end{figure}
\FloatBarrier

La tabella \ref{tab:anova piccoli aggregati} indica che entrambi i
fattori sono statisticamente significativi. Avendo verificato
l'assenza di interazioni tra \emph{Conduzione} e \emph{Lavorazione} è
quindi stato effettuato un test di Tukey-Siegel.



% latex table generated in R 3.4.2 by xtable 1.8-2 package
% Sun Nov  5 14:09:38 2017
\begin{table}[ht]
\centering
\caption{ Tabella ANOVA per i valori di densità misurati con il metodo
  \emph{Clod}}
\label{tab:anova piccoli aggregati}
\begin{tabular}{lrrrrr}
  \hline
 & Df & Sum Sq & Mean Sq & F value & Pr($>$F) \\ 
  \hline
Conduzione & 1 & 0.03 & 0.03 & 6.31 & 0.0135 \\ 
  Lavorazione & 2 & 0.05 & 0.02 & 5.83 & 0.0040 \\ 
  Totale & 104 & 0.42 & 0.00 &  &  \\ 
   \hline
\end{tabular}
\end{table}




% latex table generated in R 3.4.2 by xtable 1.8-2 package
% Sun Nov  5 14:09:38 2017
\begin{table}[ht]
\centering
\caption{Valori medi della densità apparente (in
  \SI{}{\gram\per\cubic\centi\metre}), relativi al tipo di conduzione
  e lavorazione }
\label{tab:summaryClod}
\begin{tabular}{llcccc}
  \hline
Conduzione & Lavorazione & Media & Dev. std & n & Tukey \\ 
  \hline
Convenzionale & Arato & 1.93 & 0.06 &  18 & b \\ 
   & Frangizollato & 1.89 & 0.05 &  18 & ab \\ 
   & Rippato & 1.90 & 0.06 &  18 & ab \\ 
\\
  Organico & Arato & 1.90 & 0.07 &  18 & ab \\ 
   & Frangizollato & 1.84 & 0.06 &  18 & a \\ 
   & Rippato & 1.87 & 0.07 &  18 & ab \\ 
   \hline
\end{tabular}
\end{table}

% % latex table generated in R 3.4.2 by xtable 1.8-2 package
% % Sun Nov  5 14:09:38 2017
% \begin{table}[ht]
% \centering
% \caption{tabella dei contrasti del modello adattato per i valori misurati con il metodo \emph{Clod}} 
% \label{tab:sommario piccoli aggregati}
% \begin{tabular}{lrrrr}
%   \hline
%  & Estimate & Std. Error & t value & Pr($>$$|$t$|$) \\ 
%   \hline
% CO Arato & 1.93 & 0.01 & 157.38 & 0.00 \\ 
%   Scostamento OO & -0.03 & 0.01 & -2.51 & 0.01 \\ 
%   Scostamento Frangizollato & -0.05 & 0.02 & -3.39 & 0.00 \\ 
%   Scostamento Rippato & -0.03 & 0.02 & -2.04 & 0.04 \\ 
%    \hline
% \end{tabular}
% \end{table}


\section{Dinamica della distribuzione della dimensione degli
  aggregati in seguito a disgregazione meccanica}

La misura \`e stata effettuata sia su aggregati essiccati all'aria che
su aggregati inumiditi, per studiare il fenomeno dello \emph{slaking}.
Di conseguenza sono stati adattati due modelli: uno per i campioni
inumiditi (\emph{WET}) e uno per i campioni secchi (\emph{DRY}).


La selezione del modello migliore per la sintesi dei risultati ha
portato alla seguente formula:

\nopagebreak
\begin{modello}
  \bm{Y} \sim \bm{\mu}  \oplus \bm{\beta_1} \odot x_1 \oplus \bm{\beta_2} \odot x_2 \oplus
  \bm{\beta_3} \odot x_2^2 \oplus \bm{\epsilon}
  \label{mod:Stabilita}
\end{modello}

Dove:

\nopagebreak
\begin{tabular}{lp{12cm}}
  $\bm{Y}$  & = Composizione (tre componenti: \emph{Macro, Meso, Microaggregati});\\
  $\bm{\mu}$     & = media generale;\\
  $\bm{\beta_1}$ & = coefficiente della variabile \emph{conduzione}, due livelli: \emph{Convenzionale, Biologico};\\
  $\bm{\beta_2}$  & = coefficiente della variabile \emph{tempo};\\
  $\bm{\beta_3}$ & = coefficiente della variabile \emph{tempo} elevata al quadrato;\\
  $\bm{\epsilon}$ & = residui o errore.
\end{tabular}
 
\vspace*{3em}

la quale esclude la significatività del fattore \emph{Lavorazione} .





% latex table generated in R 3.4.2 by xtable 1.8-2 package
% Sun Nov  5 14:09:40 2017
\begin{table}[hb]
\centering
\caption{Analisi della varianza relativa ai dati composizionali per la
  stabilità degli aggregati secchi all'aria (DRY nel testo)}
\label{tab:anova_compWET}
\begin{tabular}{lrrrrrr}
  \hline
 & Df & Pillai & approx F & num Df & den Df & Pr($>$F) \\ 
  \hline
Intercetta & 1 & 0.92 & 4706.24 & 2 & 835 & $<$10\verb|^|-3 \\ 
  Conduzione & 1 & 0.06 & 25.02 & 2 & 835 & $<$10\verb|^|-3 \\ 
  Tempo & 1 & 0.92 & 4525.72 & 2 & 835 & $<$10\verb|^|-3 \\ 
  Tempo\verb|^|2 & 1 & 0.19 & 97.88 & 2 & 835 & $<$10\verb|^|-3 \\ 
  Totale & 836 &  &  &  &  &  \\ 
   \hline
\end{tabular}
\end{table}


% latex table generated in R 3.4.2 by xtable 1.8-2 package
% Sun Nov  5 14:09:40 2017
\begin{table}[ht]
\centering
\caption{Analisi della varianza relativa ai dati composizionali per la
  stabilità degli aggregati previamente inumiditi (WET nel testo)}
\label{tab:anova_compDRY}
\begin{tabular}{lrrrrrr}
  \hline
 & Df & Pillai & approx F & num Df & den Df & Pr($>$F) \\ 
  \hline
Intercetta & 1 & 0.98 & 18863.73 & 2 & 857 & $<$10\verb|^|-3 \\ 
  Conduzione & 1 & 0.10 & 49.27 & 2 & 857 & $<$10\verb|^|-3 \\ 
  Tempo & 1 & 0.94 & 6718.41 & 2 & 857 & $<$10\verb|^|-3 \\ 
  Tempo\verb|^|2 & 1 & 0.19 & 101.99 & 2 & 857 & $<$10\verb|^|-3 \\ 
  Totale & 858 &  &  &  &  &  \\ 
   \hline
\end{tabular}
\end{table}


In entrambe le condizioni, WET e DRY, la dinamica di distruzione degli
aggregati col passare del tempo (visibile nel diagramma ternario in
Fig. \ref{fig:composiz_stabilita}) è diversa per le due umidità per
quanto riguarda pendenze, intercette e curvature. Alla fine del
processo distruttivo le dimensioni degli aggregati sono verosimilmente
le stesse.

Ma all'interno della categoria (WET o DRY), la dinamica è invariante
rispetto alla \emph{Conduzione}, tranne che per i punti iniziali, il
tempo zero, nei quali i campioni \emph{Convenzionali} mostrano
aggregati di maggiori dimensioni in maggiore percentuale. Questo
comportamento è visibile in Fig. \ref{fig:composiz_stabilita} e
confermato formalmente dalle tabelle \ref{tab:anova_compWET}
e \ref{tab:anova_compDRY} (alta significatività della seconda riga,
che indica la differenza di intercetta tra le due conduzioni).




% latex table generated in R 3.4.2 by xtable 1.8-2 package
% Sun Nov  5 14:09:40 2017
\begin{table}[ht]
\centering
\caption{Distribuzione dei frammenti di particelle degli aggregati
  inumiditi in seguito a: i) inizio misura (immersione): ii) inizio
  sonicatura iii) fine ciclo.}
\label{tab:iufw}
\begin{tabular}{lllll}
  \hline
  Conduzione & Fase & Macro (\%) & Meso (\%) & Micro (\%) \\
             &      & >250 um   & tra 20 e 250 um & < 20 um\\
  \hline
Convenzionale & inizio misura & 24.5 & 69.4 & 6 \\ 
    & inizio sonicatura & 9.3 & 63.6 & 27.1 \\ 
    & fine misura & 5.2 & 29.7 & 65.1 \\ 
\\ 
  Biologico  & inizio misura & 20.3 & 72.9 & 6.8 \\ 
    & inizio sonicatura & 7.3 & 63.6 & 29.1 \\ 
    & fine misura & 4 & 28.7 & 67.4 \\ 
   \hline
\end{tabular}
\end{table}

% latex table generated in R 3.4.2 by xtable 1.8-2 package
% Sun Nov  5 14:09:40 2017
\begin{table}[ht]
\centering
\caption{Distribuzione dei frammenti di particelle degli aggregati
  essiccati in seguito a: i) inizio misura (immersione): ii) inizio
  sonicatura iii) fine ciclo }
\label{tab:iufd}
\begin{tabular}{lllll}
  \hline
  Conduzione & Fase & Macro (\%) & Meso (\%) & Micro (\%) \\
             &      & >250 um   & tra 20 e 250 um & < 20 um\\
  \hline
Convenzionale & inizio  misura & 47.6 & 44.2 & 8.1 \\ 
    & inizio sonicatura & 10.8 & 54.9 & 34.3 \\ 
    & fine misura & 5 & 31.2 & 63.9 \\ 
 \\ 
  Biologico  & inizio misura & 42.6 & 47.9 & 9.5 \\ 
    & inizio sonicatura & 8.8 & 54.3 & 36.8 \\ 
    & fine misura & 3.9 & 29.8 & 66.3 \\ 
   \hline
\end{tabular}
\end{table}



\begin{figure}[ht]
  \centering
\includegraphics{Tesi_GIT-figboh3}

\caption[Evoluzione della distribuzione granulometrica delle
particelle in seguito alla distruzione degli aggregati]{Evoluzione
  della distribuzione granulometrica delle particelle liberatisi in
  seguito a: \newline i) inizio misura (immersione, punti indicati
  dalle lettere minuscole \emph{c} e \emph{o}) \newline ii) accensione
  sonicatore (lettere \emph{W} e \emph{D}) \newline iii) fine ciclo.
  \newline Gli apici del triangolo rappresentano particelle maggiori
  di 250 (MACRO), comprese tra 250 e 20 (MESO) e inferiori a
  \SI{20}{\micro\metre} (MICRO). \newline I rombi rossi (OO) e neri
  (CO) indicano la distribuzione granulometrica determinata per
  densimetria. \newline I punti iniziali, contrassegnati dalle lettere
  C e O, indicano una maggiore stabilità degli aggregati provenioenti
  dalle parcelle \emph{Convenzionali} }
  \label{fig:composiz_stabilita}
\end{figure}
\FloatBarrier






\section{Distribuzione dimensionale dei pori}



I dati ottenuti sono stati analizzati mediante tecniche di analisi
composizionale. 

 
Il modello composizionale adattato per i dati linearizzati è il seguente:
\nopagebreak
\begin{modello}
  \bm{Y} \sim \bm{\mu}  \oplus \bm{\beta_1} \odot x_1 \oplus
  \bm{\beta_2} \odot x_2 \oplus  \bm{\epsilon}
  \label{mod:Porosita}
\end{modello}

Dove:

\nopagebreak
\begin{tabular}{lp{12cm}}
  $\bm{Y}$  & = Composizione (tre componenti: pori \emph{residui},
              pori di \emph{trasmissione} e pori di \emph{immagazzinamento});\\
  $\bm{\mu}$     & = media generale;\\
  $\bm{\beta_1}$ & = coefficiente della variabile \emph{conduzione}, due livelli: \emph{Convenzionale, Biologico};\\
  $\bm{\beta_2}$  & = coefficiente della variabile \emph{lavorazione},
                    due livelli: \emph{arato, rippato, frangizollato};\\
  $\bm{\epsilon}$ & = residui o errore.
\end{tabular}
 
\vspace*{3em}

In Fig. \ref{fig:Poros_triangolo} e in tabella
\ref{tab:poros_anova} sono visibili i valori medi della composizione
per ogni combinazione di trattamenti.

% latex table generated in R 3.4.2 by xtable 1.8-2 package
% Sun Nov  5 14:09:40 2017
\begin{table}[hb]
\centering
\caption{Sommario della distribuzione dei pori nelle tre classi
  dimensionali dei dati ottenuti dalla analisi con porosimetria a
  mercurio}
\label{tab:Poro_medie}
\begin{tabular}{llccc}
  \hline
  Conduzione & Lavorazione & Residui   & Immagazzinamento & Trasmissione \\
             &             &  < 0.5 um &  tra 0.5 e 50 um & > 50 um      \\
             &             &     (\%)  &            (\%)  &  (\%) \\
  \hline
  Convenzionale & Arato & 74 & 24 & 2 \\
  & Rippato & 61 & 34 & 6 \\
  & Frangizollato & 65 & 33 & 2 \\
\\
  Biologico & Arato & 56 & 41 & 3 \\
  & Rippato & 56 & 41 & 3 \\
  & Frangizollato & 64 & 32 & 4 \\
  \hline
\end{tabular}
\end{table}% latex table generated in R 3.4.2 by xtable 1.8-2 package
% Sun Nov  5 14:09:40 2017

\begin{table}[ht]
\centering
\caption{ANOVA del modello composizionale dei dati ottenuti dalle
  analisi con porosimetria a mercurio }
\label{tab:poros_anova}
\begin{tabular}{lrrrrrr}
  \hline
 & Df & Pillai & approx F & num Df & den Df & Pr($>$F) \\ 
  \hline
Intercetta & 1 & 0.94 & 60.45 & 2 & 7 & $<$10\verb|^|-3 \\ 
  Conduzione & 1 & 0.14 & 0.58 & 2 & 7 & 0.585 \\ 
  Lavorazione & 2 & 0.52 & 1.41 & 4 & 16 & 0.275 \\ 
  Totale & 8 &  &  &  &  &  \\ 
   \hline
\end{tabular}
\end{table}

\`E possibile evincere come i dati non permettano di evidenziare delle
differenze. I punti all'interno del diagramma ternario si trovano in
corrispondenza dei pori residui (minori di 0.5 \SI{}{\micro\metre}),
indicando quindi una prevalenza di questi pori rispetto ai pori di
trasmissione e immagazzinamento.

 
% \noindent%
% \begin{minipage}{\linewidth}% to keep image and caption on one page
%   \makebox[\linewidth]{%        to center the image
\begin{figure}[ht]
  \centering
\includegraphics{Tesi_GIT-figurina}
\caption[didascalia corta]{Diagramma ternario della distribuzione dei
  pori. Gli apici del triangolo rappresentano le classi dimensionali
  in cui i pori sono divisi in: inferiori di 0.5 (pori residui),
  comprese tra 0.5 e 50 (pori di immagazzinamento) e superiori di 50
  \SI{}{\micro\metre} (pori di trasmissione)}
  \label{fig:Poros_triangolo}
\end{figure}
% \end{minipage}
\FloatBarrier

La tabella \ref{tab:poros_anova}, mostra che all'interno dei
trattamenti non sono presenti differenze statisticamente
significative. Di conseguenza, dato il basso numero di campioni, non
\`e possibile evidenziare una significatività dei trattamenti.

Considerato che le tecniche di statistica composizionale non
considerano la somma totale delle composizioni ma i loro rapporti, \`e
stata tentata una analisi statistica univariata prendendo come
variabile dipendente il valore della porosità totale del campione,
normalizzato al suo peso.


% latex table generated in R 3.4.2 by xtable 1.8-2 package
% Sun Nov  5 14:09:40 2017
\begin{table}[ht]
\centering
\caption{ANOVA del modello adattato alla porosità totale} 
\label{tab:tot_anova}
\begin{tabular}{lrrrrr}

  \hline
 & Df & Sum Sq & Mean Sq & F value & Pr($>$F) \\ 
  \hline
Conduzione & 1 & 1562.18 & 1562.18 & 2.37 & 0.1496 \\ 
  Lavorazione & 2 & 3129.81 & 1564.91 & 2.38 & 0.1352 \\ 
  Totale & 12 & 7906.89 & 658.91 &  &  \\ 
   \hline
\end{tabular}
\end{table}

La tabella Anova \ref{tab:tot_anova} ci permette di affermare che il
numero di campioni non \`e sufficiente per determinare la
significatività dei trattamenti, tuttavia come si può vedere dalla
tabella \ref{tab:tot_sommario} , i dati tenderebbero a confermare il
risultato dell'analisi della densità apparente misurata con il metodo
\emph{Clod} dal quale risultava una maggior porosità nei campioni di
suolo provenienti da appezzamenti condotti a \emph{Biologico}.

\begin{table}[ht]
\centering
\caption{Sommario della porosit\`a totale dei dati ricavati dalle analisi della porosimetria a mercurio} 
\label{tab:tot_sommario}
\begin{tabular}{llcccc}
  \hline
Conduzione & Lavorazione & Porosità totale & ST.DEV & n & Tukey \\ 
  \hline
Convenzionale & Arato & 159.91 & 16.21 &   2 & a \\ 
   & Rippato & 168.37 & 20.92 &   3 & a \\ 
   & Frangizollato & 186.13 & 30.38 &   3 & a \\ 
   &  &  &  &  &  \\ 
  Biologico & Arato & 168.03 & 9.62 &   2 & a \\ 
   & Rippato & 192.20 & 38.54 &   4 & a \\ 
   & Frangizollato & 218.28 & 7.29 &   2 & a \\ 
   \hline
\end{tabular}
\end{table}

% latex table generated in R 3.4.2 by xtable 1.8-2 package
% Sun Nov  5 14:09:40 2017
% \begin{table}[ht]
% \centering
% \caption{Tabella di contingenza del modello adattato alla porosità totale} 
% \label{tab:tot_sommario}
% \begin{tabular}{lrrrr}
%   \hline
%  & Estimate & Std. Error & t value & Pr($>$$|$t$|$) \\ 
%   \hline
% Convenzionale Arato & 152.787 & 14.381 & 10.624 & $<$10\verb|^|-3 \\ 
%   Biologico Arato & 22.367 & 12.974 & 1.724 & 0.11 \\ 
%   Convenzionale Rippato & 16.417 & 16.116 & 1.019 & 0.328 \\ 
%   Convenzionale F.zollato & 37.258 & 17.268 & 2.158 & 0.052 \\ 
%    \hline
% \end{tabular}
% \end{table}


\section{Penetrometria}

I dati penetrometrici (0-80 cm) sono stati analizzati da diversi punti
di vista: medie, modelli misti con funzioni sigmoidali e infine come dati spettrali.

I primi due metodi non hanno fornito risposte convincenti, mentre
considerando i dati di resistenza alla penetrazione (anni 2015 et
2016) come se fossero degli spettri e attraverso PCA, si ottengono i
risultati mostrati in Fig. \ref{fig:PenetrometriaLav} e
\ref{fig:PenetrometriaCond}


\begin{figure}[ht]
  \centering
\includegraphics[page=1, width=\textwidth]{../grafici/PCA_penetrometria_LavorazETConduz}
\caption{Di}
  \label{fig:PenetrometriaLav}
\end{figure}

\begin{figure}[ht]
  \centering
\includegraphics[page=2, width=\textwidth]{../grafici/PCA_penetrometria_LavorazETConduz}
\caption{}
  \label{fig:PenetrometriaCond}
\end{figure}



%%% Local Variables:
%%% mode: latex
%%% TeX-master: t
%%% End:
\end{document}