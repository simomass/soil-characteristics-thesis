\documentclass[a4paper]{article}

\usepackage[english]{babel}  % force American English hyphenation patterns
\usepackage[utf8x]{inputenc}  % unicode
\usepackage{graphicx}
\usepackage{wrapfig}
\usepackage{lipsum}  % generates filler text

\title{Package Example: wrapfig}
\author{writeLaTeX}

\begin{document}
Nell’anno agrario
 2016/2017 nel campo OO2 `e stato coltivato il Girasole, variet`a
 Solaris non trattato, usando una seminatrice di precisione con una
 densit`a di 75 000 piante ad ettaro. La coltura precedente era la
 lenticchia, variet`a Val di Nevola. Le lavorazioni per preparare il
 letto di semina sono state tutte svolte l’8 Settembre 2016 in base
 alla parcella, ovvero l’aratura nelle parcelle A, la rippatura nelle
 parcelle B e la frangizzolatura nelle C.  ven 29 set, 2017 11:46:10
 Tesi Dalila Pasquini Materiali e metodi ura 3.3: Appezzamenti a
 biologico a sinistra e convenzionale a destra. I plot sono icati con
 codici alfanumerici identificativi: OO seguito da un numero indica
 ppezzamento biologico e CO seguito dal numero rappresenta invece il
 convenzionale; lettera maiuscola indica la lavorazione A (aratura), B
 (rippatura), C angizollatura); il numero (1,2,3) a pedice differenzia
 le repliche della lavorazione; lettere minuscole (a, m, b,) stanno ad
 indicare i tre campioni all’interno di ogni golo plot .  tata poi
 effettuata una concimazione con pollina il 22 Febbraio 2017 e il ´
 980 giorno successivo `e stata fatta una frangizzolatura. Il 29 Marzo
 il campo `e stato erpicato e il giorno successivo `e avvenuta la
 semina.  Lo stesso anno, nel campo OO4, precedentemente coltivato a
 cece, va- riet`a Pasci`a , `e stato coltivato l’orzo, variet`a
 Campagne non trattato, con una seminatrice a righe. Anche qui la
 preparazione del letto di semina, in base alla parcella, `e stata
 effettuata l’8 Settembre 2016. C’`e stata una fran- gizzollatura il 5
 Dicembre con successiva semina, dando 190 Kg di seme ad ettaro, ed il
 15 Marzo del 2017 `e stata fatta una strigliatura.  Nel campo CO09
 invece `e stato seminato l’orzo distico variet`a Campagne trattato
 con Redico puro all’8.7%. La coltura precedente era il girasole,
 variet`a Solaris trattato con Apron-xl p.a metalaxil-m.  L’8
 Settembre 2016 sono state effettuate le lavorazioni che distinguono
 le differenti parcelle, poi una prima concimazione il 5 Dicembre
 dando 192 i Dalila Pasquini ven 29 set, 2017 11:46:10 57 Materiali e
 metodi Kg/ha di fosfato biammonico (18:46), ed a seguire `e stata
 effettuata una frangizzolatura e la semina con una seminatrice a
 righe, dando 190 Kg/ha di seme. Il 15 Marzo `e stata fatta una
 seconda concimazione con nitrato d’ammonio (150 Kg/ha) e due
 settimane dopo c’`e stato il diserbo usando un mix composto da Axial
 (1 L) + Axial Proto (250 mL) + Logran(55 g) ad ettaro. L’11 Aprile `e
 stata poi distribuita l’urea con una dose di 150 Kg 0 a ettaro.  1
 Infine, nel campo CO10, precedentemente coltivato ad orzo distico
 Cam- 2 pagne trattato con Redico, `e stato seminato il Girasole,
 variet`a Solaris trat- 3 tato con Apron-xl p.a metalaxil-m. Le
 lavorazioni iniziali sono state fatte 4 sempre ad inizio autunno,
 nella stessa data degli altri campi e a seguire una 5
 frangizzollatura il 23 Febbraio e poi un’erpicatura il 29 Marzo. La
 semina `e 6 stata fatta il 30 Marzo, con una densit`a di 75 000
 piante ad ettaro usando 7 una seminatrice di
 precisione. Contemporaneamente alla semina `e stata fat- 8 ta una
 concimazione con 20:10:10 (150 Kg/ha) e un diserbo a fascia usando 9
 500 mL del prodotto GOAL.

\end{document}
