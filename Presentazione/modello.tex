\ifdefined\ishandout
\documentclass[xcolor={usenames, table, x11names}, handout, 10pt]{beamer} 
\usepackage{pgfpages}
\pgfpagesuselayout{2 on 1}[a4paper,border shrink=5mm]
\else
\ifdefined\isNote
\documentclass[xcolor={usenames, table, x11names}, final, 10pt]{beamer} 
\setbeameroption{show only notes}
\else
\ifdefined\isDraft
\documentclass[xcolor={usenames, table, x11names}, draft, 10pt]{beamer} 
\else
\documentclass[xcolor={usenames, table, x11names}, final, 10pt]{beamer} 
\fi
\fi
\fi


% final/draft cambia velocità compilazione. Anche "handout" è ammesso
\usepackage{etex}% serve per caricare dopo tikz, richiesto da chemfig
% http://tex.stackexchange.com/questions/7896/no-room-for-a-new-dimen-when-including-tikz
\setbeamercolor{emph}{fg=red}
\renewcommand<>{\emph}[1]{%
  {\usebeamercolor[fg]{emph}\only#2{\itshape}#1}%
}
\mode<presentation>
{%
  \usetheme[]{Warsaw} %Berlin
  % [compress] lascia una riga coi pallini sola in cima
  \usecolortheme{seahorse}% crane=giallo, beetle: il bianco/nero in alto non si legge 
  % wolverine_mistogialloblu
  % \usefonttheme[onlylarge]{structuresmallcapsserif}
  % \usefonttheme[onlysmall]{structurebold}
  \setbeamercolor{emph}{fg=red}
  \setbeamerfont{title}{shape=\itshape, family=\rmfamily}
  % \setbeamercolor{title}{fg=red!80!black}
  % \setbeamercolor{title}{fg=red!80!black,bg=red!20!white}
  \setbeamertemplate{footline}[frame number]{}
  % \setbeamertemplate{navigation symbols}[vertical]
  % la riga sotto toglie le iconcine di navigazione in basso che non funzionano
  \setbeamertemplate{navigation symbols}{} 
  \setbeamercovered{transparent}
  \useinnertheme{rounded}% mette le palline al posto dei quadrati
  \setbeamertemplate{blocks}[rounded][shadow=true] %aggiunge l'ombra ai blocchi
  % \setbeamercolor{background canvas}{bg=black!20}
  \setbeamercolor{section in head/foot}{fg=black, bg=blue!10}
  \setbeamertemplate{section in head/foot shaded}[default][40]
  %% le due righe sopra cambiano il colore della striscia in alto, la headline
  \setbeamertemplate{headline}{%
    \leavevmode%
    \hbox{%
      \begin{beamercolorbox}[wd=\paperwidth,ht=5ex,dp=1ex]{section in head/foot}%
        \insertsectionnavigationhorizontal{\paperwidth}{}{\hskip0pt plus1filll}
        \insertsubsectionnavigationhorizontal{\paperwidth}{\hskip0pt plus1filll}{}
      \end{beamercolorbox}%
    }
  }
}%% fine del mode presentation
\mode<handout>
{%
  % Questo pezzo makeatother serve per mettere le pagine e il nome sul
  % pie' di pagina
  \makeatother
  \setbeamertemplate{footline}
  {
    \leavevmode%
    \hbox{%
      \begin{beamercolorbox}[wd=.2\paperwidth,ht=2.25ex,dp=1ex,center]{author in head/foot}%
        \usebeamerfont{author in head/foot}\insertshortauthor
      \end{beamercolorbox}%
      \begin{beamercolorbox}[wd=.8\paperwidth,ht=2.25ex,dp=1ex,center]{title in head/foot}%
        \usebeamerfont{title in head/foot}
        Chimica del suolo. Scienze vivaistiche, ambiente e gestione del
        verde\hspace*{3em}
        \input{AnnoAccademico.txt} pag.
        \insertframenumber{} %attiva questo per i numeri di pagina ma fa casino 
      \end{beamercolorbox}}%
    \vskip0pt%
  }
  \makeatletter
  \setbeamercolor{background canvas}{bg=black!5}
}
\usepackage[polutonikogreek,italian]{babel} % parole in greco antico polutonikogreek (lez 01)
\usepackage[utf8x]{inputenc}
% http://zyliu2005.blogspot.com/2008/08/latex-change-mrow-color-of-table.html
% se non si mette "xcolor" in questo modo, pdflatex fa casino
\usepackage[x11names]{xcolor} % richiesto per colortbl
\usepackage{color} % richiesto per ambiente picture linee a colori
\usepackage{graphics} % legge i file pdf jpg jpeg png
\usepackage{siunitx}% serve per \celsius e le unita di misura
\DeclareSIQualifier\azoto{\ce{N2}}
\DeclareSIQualifier\ammoniaca{\ce{NH3}}
\DeclareSIQualifier\ossigeno{\ce{O2}}
\DeclareSIUnit\anno{anno}
\DeclareSIUnit\cal{cal}
\DeclareSIUnit\secolo{secolo} 
\DeclareSIUnit\vol{v}
% fine delle unità di misura speciali
% \usepackage[final]{pdfpages} % per mettere pdf a pagine multiple (lez 01)
\usepackage[version=4]{mhchem} % 26 feb 2015: mail con martin hensel
% necessario mchem.sty da lui inviato per \ce{Na, Ca, Ba,} le virgole

\usepackage{xymtexpdf}
\usepackage{chmst-pdf}

% \usepackage{xymtex}
% \usepackage[chemist]{chemtimes}
\usepackage{xmpmulti} %serve per overlay dei grafici
\usepackage{graphicx}
% \usepackage{booktabs} % \toprule \midrule \bottomrule nelle tabelle (lez 01)
\usepackage{multirow} % righe in tabelle 
\usepackage{tabularx}
\usepackage{array} % richiesto per colorare e definire colonne e per
% dcolumn
\usepackage{dcolumn} % serve per allineare le colonne sulle virgole
\usepackage{ulem} % sottolineature e overstrike
\usepackage{chemfig, xstring} % chiama tikz xstring lez 11
% \usepackage{gensymb} % serve per \celsius dopo passaggio a wheezy (lez 01)
% \usepackage{rotating}% per ruotare figure
\usepackage{hyperref} %serve al pacchetto multimedia sotto
\usepackage{multimedia} % per animazioni in file esterni
% \usepackage{media9} % per animazioni in file esterni
% http://latex-community.org/forum/viewtopic.php?f=45&t=20581
% \usepackage{amsmath} % serve per le equazioni anche chimiche 

\usepackage{multicol}% per spezzare gli indici in due colonne. riga
% 135
%% \pgfdeclareimage[height=1cm]{logo}{DIPSA_BN.eps}
%% \logo{\pgfuseimage{logo}}
% \logo{\includegraphics[width=0.5cm]{DIPSA_BN.jpg}}

\usepackage{textcomp} % serve per il segno permille \textperthousand lez 09
% \usepackage{../mol2chemfig} %%lez 11
% \usepackage{animate} % serve per le animazioni lez 11
% \usepackage{tabu} %per colorare il testo delle tabelle, non lo sfondo lez 11

% \mathversion{boldchem}
\newchemenvironment{chemalign}{align}
\newchemenvironment{chemalign*}{align*}
% \newchemenvironment{chemalignat}{alignat}
\newcolumntype{,}{D{,}{,}{2}}
%% http://tex.stackexchange.com/questions/13423/how-to-change-the-color-of-href-links-for-real
\definecolor{links}{HTML}{FF00FF}
% FF00FF magenta, FF0000 rosso. bluastro 2A1B81
% http://www.color-hex.com/color-names.html
\hypersetup{colorlinks,linkcolor=,urlcolor=links}
%% 2 righe sopra da lez 01
% \newcommand{\greco}[1]{% per scrivere direttamente in greco
% \begin{otherlanguage*}{greek}#1\end{otherlanguage*}}
\newcommand*\circleatom[1]{\tikz\node[circle,draw,fill=green!30]{\printatom{#1}};
}

\DeclareMathSizes{12}{30}{16}{12}
\title{25 anni di conduzione Biologica}% \input{NomeDelCorso.txt}}
\author{\input{autore.txt}}
\institute{\input{DISPAA.txt}}
% \date{\input{AnnoAccademico.txt}}
