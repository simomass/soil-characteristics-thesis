\documentclass[10pt]{beamer}
\usetheme[
%%% option passed to the outer theme
% progressstyle=fixedCircCnt,   % fixedCircCnt, movingCircCnt (moving is deault)
]{Feather}

% If you want to change the colors of the various elements in the theme, edit and uncomment the following lines

% Change the bar colors:
% \setbeamercolor{Feather}{fg=red!20,bg=red}

% Change the color of the structural elements:
% \setbeamercolor{structure}{fg=red}

% Change the frame title text color:
% \setbeamercolor{frametitle}{fg=blue}

% Change the normal text color background:
% \setbeamercolor{normal text}{fg=black,bg=gray!10}

% -------------------------------------------------------
% INCLUDE PACKAGES
% -------------------------------------------------------

\usepackage[utf8]{inputenc}
\usepackage[english]{babel}
\usepackage[T1]{fontenc}
\usepackage{helvet}
\usepackage{pgfplots}
\usepackage{ragged2e}
% -------------------------------------------------------
% DEFFINING AND REDEFINING COMMANDS
% -------------------------------------------------------

% colored hyperlinks
\newcommand{\chref}[2]{
  \href{#1}{{\usebeamercolor[bg]{Feather}#2}}
}

% -------------------------------------------------------
% INFORMATION IN THE TITLE PAGE
% -------------------------------------------------------
% \setbeamertemplate{title page}
% {
\title[] % [] is optional - is placed on the bottom of the sidebar on every slide
{ % is placed on the title page
  \textbf{25 anni di conduzione Biologica in area Mediterranea}
}

\subtitle[Uno studio di fisica del suolo]
{
  \textbf{Uno studio di fisica del suolo}
}

\author[Simone Massenzio]
{ 
  Candidato: \\Simone Massenzio \\
  Relatore: \\Dott. Ottorino Luca Pantani\\
  Correlatori:\\
  Dott. Luigi-Paolo D'acqui, Prof. Gaio Cesare Pacini}     



\institute[] { \emph{Dipartimento di Scienze della Produzioni Animali e
    dell'Ambiente\\
    Universit\`a degli studi di Firenze - UniFI\\}
  
  % there must be an empty line above this line - otherwise some
  % unwanted space is added between the university and the country (I
  %   % do not know why;( )
}

\date{\today}

% -------------------------------------------------------
% THE BODY OF THE PRESENTATION
% -------------------------------------------------------
\setbeamercovered{transparent}


\begin{document}
{\1
  \begin{frame}[noframenumbering]%{\footnotesize{Dipartimento di Scienze della Produzioni Animali e
    % dell'Ambiente\\
    % Universit\`a degli studi di Firenze - UniFI}}
    \titlepage
  \end{frame}}



% presumo che queste righe mettano dei segnali in ogni parte e sezione
% \AtBeginPart{\frame<beamer>{\partpage
% \transsplitverticalout[duration=1] 
% \begin{block}{}
%       %   \tableofcontents[subsectionstyle=show]
%   \tableofcontents[subsectionstyle=hide]
% \end{block}
% }}

%   \AtBeginSection[]{\frame<beamer>{%
%   \begin{block}{}
%     \tableofcontents[currentsection, subsectionstyle=hide]
%   \end{block}
% }
% }



\begin{frame}{Obiettivi}{Pippo}
  % \frametitle{Obiettivi}
  \large
  \begin{itemize}[<+->]
  \item verificare la presenza di differenze nelle caratteristiche
    fisiche del suolo in aree: coltivate convenzionalmente (\emph{CO})
    oppure coltivate a biologico (\emph{OO}) tramite:
    \begin{itemize}
    \item Densit\`a apparente
      \begin{itemize}
      \item Metodo \emph{Core}
      \item Metodo \emph{Clod}
      \end{itemize}
    \item Stabilit\`a degli aggregati tramite dinamica della
      distribuzione della dimensione degli aggregati;
    \item Distribuzione dimensionale dei pori tramite tecniche di
      porosimetria ad intrusione di mercurio;      
    \end{itemize}
  \item verificare eventuali effetti dovuti a diverse lavorazioni
    (aratura, rippatura, frangizollatura).
  \end{itemize}
\end{frame}

\section{Densit\`a apparente}
\subsection{Metodi}

\begin{frame}{Misura densita}{Due metodi: \emph{Core} e \emph{Clod}}
  \begin{columns}[c]
    \column{.50\textwidth}
    \onslide<1-> Metodo \emph{Core}
    \begin{itemize}[<+->] 
    \item estrazione di un volume di suolo noto mediante un cilindro
      non deformabile di ottone
    \item essiccazione del campione di terreno e sua pesatura
    \item calcolo della densit\`a apparente
      ($\rho_{ApparenteClod} = Peso/Volume$)
    \end{itemize}
    \column{.48\textwidth}
    \onslide<1->\includegraphics[width=\textwidth]{../foto/cilindroOttone.jpeg}
  \end{columns}
\end{frame}


\begin{frame}{Misura densita}{Due metodi: \emph{Core} e \emph{Clod}}

  \begin{columns}[c]
    \column{.50\textwidth}
    \onslide<1-> Metodo \emph{Clod}
    \begin{itemize}[<+->]
    \item separazione di un aggregato (3-5 cm di diametro) dal
      campione di suolo
    \item misura del peso $P_{petrolio}$ del volume di petrolio
      spostato dall'aggregato, ovvero della spinta idrostatica
    \item essiccazione a $150^oC$ per una notte e pesatura campione
      secco
    \item calcolo della densit\`a apparente
      ($\rho_{ApparenteCore} = Peso/Volume$)
    \end{itemize}
    \column{.48\textwidth}
    \onslide<1->\includegraphics[width=\textwidth]{../foto/pallinespinta.jpeg}
  \end{columns}
\end{frame}

\subsection{Adattamento del modello lineare}
\begin{frame}
  \frametitle{Sintesi tramite modello lineare}
  \transwipe<4>[direction=90]
  \begin{itemize}
    \onslide<2->\item 
    \vspace{0.25cm}
    $Y \sim \mu + \beta_1x_1 + \beta_2x_2 + \beta_3x_3 + \epsilon$
    \vspace{0.25cm}
    in cui le variabili categoriche sono:
    \begin{itemize}
      \onslide<2->\item $\beta_1$ anno \newline 2015, 2016 \emph{Non
        presente nel modello riguardante i dati ricavati dal metodo \emph{Clod}}
      \onslide<2->\item $\beta_2$ conduzione \newline convenzionale
      (CO), biologico (OO) \onslide<2->\item $\beta_3$
      lavorazioni \newline arato, rippato, frangizollato
      \onslide<2->\item$\epsilon$ residui o errore
    \end{itemize}
    \onslide<3->\item validazione del modello attraverso l'analisi
    dei residui 
    \onslide<4->\item analisi della varianza (ANOVA)
  \end{itemize}
\end{frame}

% \subsection{Risultati Core}
\begin{frame}{Sommario dati metodo \emph{Core}}
  \footnotesize
  % latex table generated in R 3.4.0 by xtable 1.8-2 package
  % Thu Jun 22 15:19:10 2017
  \begin{table}[ht]
    \centering
    \begin{tabular}{lllccc}
      \hline
      Anno & Conduzione & Lavorazione & Densit\`a apparente
                                        ($g/cm^3$) & dev std & n \\ 
      \hline
      2015 & Convenzionale & Arato & 1.35 & 0.14 &   5 \\ 
           &   & Frangizollato & 1.34 & 0.07 &   5 \\ 
           &   & Rippato & 1.38 & 0.10 &   5 \\ 
           & Organico & Arato & 1.38 & 0.07 &   5 \\ 
           &   & Frangizollato & 1.42 & 0.13 &   5 \\ 
           &   & Rippato & 1.43 & 0.04 &   5 \\ 
      \\
      2016 & Convenzionale & Arato & 1.38 & 0.16 &   6 \\ 
           &   & Frangizollato & 1.35 & 0.15 &   6 \\ 
           &   & Rippato & 1.26 & 0.09 &   5 \\ 
           & Organico & Arato & 1.37 & 0.06 &   6 \\ 
           &   & Frangizollato & 1.35 & 0.07 &   6 \\ 
           &   & Rippato & 1.40 & 0.15 &   6 \\ 
      \hline
    \end{tabular}
    \label{tab:RiassuntoDensitaCAmpo}
  \end{table}
\end{frame}

\begin{frame}
  \begin{figure}[ht]
    \includegraphics[width=0.8\textwidth]{../tesi/Tesi_GIT-figboh2.pdf}
  \end{figure}
\end{frame}

\begin{frame}{Tabella ANOVA per i valori di densità rilevati col metodo \emph{Core}} 
  % latex table generated in R 3.4.0 by xtable 1.8-2 package
  % Thu Jun 22 16:16:27 2017
  \begin{table}[ht]
    \centering
    \label{tab:anova del modello}
    \begin{tabular}{lrrrrr}
      \hline
      & Df & Sum Sq & Mean Sq & F value & Pr($>$F) \\ 
      \hline
      Anno & 1 & 0.02 & 0.02 & 1.41 & 0.2390 \\ 
      Conduzione & 1 & 0.04 & 0.04 & 3.29 & 0.0745 \\ 
      Lavorazione & 2 & 0.00 & 0.00 & 0.03 & 0.9728 \\ 
      residui & 60 & 0.71 & 0.01 &  &  \\ 
      \hline
    \end{tabular}
  \end{table}
\end{frame}

% \begin{frame}
%   \begin{figure}[ht]
%     \includegraphics[width=0.8\textwidth]{../grafici/density/plot_2anni.pdf}
%   \end{figure}
% \end{frame}

% \begin{frame}{t table densit\`a grandi volumi}
%   %   latex table generated in R 3.4.0 by xtable 1.8-2 package
%   %   Fri Jun 23 11:16:03 2017
%   \begin{table}[ht]
%     \centering
%     \begin{tabular}{rrrrr}
%       \hline
%       &Estimate & Std. Error & t value & Pr($>$$|$t$|$) \\ 
%       \hline
%       Convenzionale 2015 & 1.36  & 0.02 & 62.28 & 0.00 \\ 
%       Convenzionale 2016 & -0.03 & 0.02 & -1.52 & 0.13 \\ 
%       Biologico 2015     &  0.05  & 0.02 &  2.18 & 0.03 \\ 
%       \hline
%     \end{tabular}
%     \label{tab:Riassunto densit`a campo 2015 e 2016}
%   \end{table}
%   \begin{block}{\emph{Scostamenti!}}
%     La prima riga riporta il valore del trattamento \emph{corner}
%     (1.36 $g/cm^3$).

%     Le altre righe riportano invece il valore dello scostamento (e
%     relativo t-test) rispetto al trattamento \emph{corner}.

%     Di conseguenza il trattamento biologico 2016 ha come valore assoluto
%     1.36 - 0.03 + 0.05 =  1.38 $g/cm^3$.
%   \end{block}
% \end{frame}





% \subsection{Risultati \emph{Clod}}

\begin{frame}{Sommario dati metodo \emph{Clod}}
  \footnotesize
  \begin{table}[ht]
    \centering
    \begin{tabular}{llrccc}
      \hline
      Conduzione & Lavorazione & Media & Dev. std & n & Tukey \\ 
      \hline
      Convenzionale & Arato & 1.93 & 0.06 &  18 & b \\ 
                 & Frangizollato & 1.89 & 0.05 &  18 & ab \\ 
                 & Rippato & 1.90 & 0.06 &  18 & ab \\ 
      Organico & Arato & 1.90 & 0.07 &  18 & ab \\ 
                 & Frangizollato & 1.84 & 0.06 &  18 & a \\ 
                 & Rippato & 1.87 & 0.07 &  18 & ab \\ 
      \hline
    \end{tabular}
    \label{tab:RiassuntoDensitaSpinta}
  \end{table}
\end{frame}

\begin{frame}
  \begin{figure}
    \includegraphics[width=0.8\textwidth]{../tesi/Tesi_GIT-figmah.pdf}
  \end{figure}
\end{frame}

\begin{frame}{Tabella ANOVA per i valori di densità rilevati col metodo \emph{Clod}}
  % latex table generated in R 3.4.0 by xtable 1.8-2 package
  % Thu Jun 22 16:32:35 2017
  \begin{table}
    \centering
    \begin{tabular}{llcccc}
      \hline
      & Df & Sum Sq & Mean Sq & F value & Pr($>$F) \\ 
      \hline
      Conduzione & 1 & 0.03 & 0.03 & 6.31 & 0.012 \\ 
      Lavorazione & 2 & 0.05 & 0.02 & 5.83 & 0.004 \\ 
      Residui & 104 & 0.42 & 0.00 &  &  \\ 
      \hline
    \end{tabular}
    \label{tab:Anova densita per spinta}
  \end{table}
\end{frame}

% \begin{frame}
%   \begin{figure}[ht]
%     \includegraphics[width=0.8\textwidth]{../grafici/density/plot_petrolio.pdf}
%   \end{figure}
% \end{frame}

% \begin{frame}{t-table  densit\`a piccoli aggregati}
%   %   latex table generated in R 3.4.0 by xtable 1.8-2 package
%   %   Thu Jun 22 10:44:00 2017
%   \footnotesize
%   %   latex table generated in R 3.4.0 by xtable 1.8-2 package
%   %   Thu Jun 22 16:31:31 2017
%   \begin{table}[ht]
%     \centering
%     \begin{tabular}{rrrrr}
%       \hline
%       & Estimate & Std. Error & t value & Pr($>$$|$t$|$) \\ 
%       \hline
%       %       Convenzionale arato & 1.9470 & 0.0295 & 65.98 & 0.0000 \\ 
%       %       Scostamento biologico & -0.0534 & 0.0295 & -1.81 & 0.0731 \\ 
%       %       Scostamento frangizollato & -0.0780 & 0.0361 & -2.16 & 0.0331 \\ 
%       %       Scostamento rippato & -0.0881 & 0.0361 & -2.44 & 0.0164 \\ 
%       Convenzionale arato & 1.93 & 0.01 & 157.38 & 0.000 \\ 
%       Scostamento biologico & -0.03 & 0.01 & -2.51 & 0.014 \\ 
%       Scostamento frangizollato & -0.05 & 0.02 & -3.39 & 0.001 \\ 
%       Scostamento rippato & -0.03 & 0.02 & -2.04 & 0.043 \\ 
%       \hline
%     \end{tabular}
%     \label{tab:Riassunto densita spinta}
%   \end{table}
% \end{frame}



% \section{Stabilit\`a degli aggregati}
% % \subsection{Strumentazione e metodo}
% \begin{frame}
%   \frametitle{Misura stabilit\`a degli aggregati}
%   %   \citep{ugolini2010basi} 
%   \begin{columns}
%     \column{.50\textwidth}
%     %     \begin{block}{}
%       \pause
%       \begin{enumerate}[<+->] 
%       \item separazione della frazione compresa tra 1 e 2 mm  
%       \item 300 mg di campione (secco a \SI{40}{\celsius}) vengono immessi
%         nel vasca di misura piena d'acqua dello strumento Hydro 2000g
%       \item acquisizione temporale della distribuzione granulometrica:
%         una al minuto per 12 minuti durante i quali \ldots
%       \item lo strumento provvede al ricircolo dell'acqua che disgrega
%         le particelle di suolo
%       \item attivazione della sonicazione dal \SI{13}{\degree} al
%         \SI{24}{\degree} minuto per completare la rottura degli
%         aggregati

%       \end{enumerate}
%     %     \end{block}
%     \column{.48\textwidth}
%     \only<3>{
%     \begin{figure}[ht]
%       \includegraphics[width=\textwidth]{../foto/Hydro.jpeg}
%     \end{figure}
%   }
%     \only<4->{
%     \begin{figure}[ht]
%       \includegraphics[width=\textwidth]{../foto/Acquisiz.jpeg}
%     \end{figure}
%   }
%   \end{columns}  

% \end{frame}

% \begin{frame}
%   \footnotesize
%   \begin{itemize}
%   \item La variazione della distribuzione nel tempo ci fornisce
%     informazioni sulla resistenza degli aggregati alle sollecitazioni
%     meccaniche.
%   \item Oltre che sui campioni \textcolor{blue}{secchi}, la misura
%     \`e stata ripetuta su aggregati 
%     \textcolor{magenta}{previamente inumiditi} per evitare lo \textit{slacking}
%   \end{itemize}

%   \begin{figure}
%     \includegraphics[width=0.6\textwidth, page=6]{../grafici/PISA2.pdf}
%   \end{figure}
% \end{frame}

% \begin{frame}
%   \begin{figure}
%     \includegraphics[width=0.9\textwidth, page=1]{../grafici/UltrasuoniSI_NO_UMIDITA_WET_DRY.pdf}
%   \end{figure}
% \end{frame}

% \begin{frame}{Tabella Anova dati composizionali}
%   \footnotesize
%   %   latex table generated in R 3.4.0 by xtable 1.8-2 package
%   %   Tue Jun 27 21:29:19 2017
%   \begin{table}
%     \centering
%     \begin{tabular}{lrrrrrr}
%       \hline
%       & Df & Pillai & approx F & num Df & den Df & Pr($>$F) \\ 
%       \hline
%       COnvenzionale & 1 & 0.92 & 4955.26 & 2 & 835 & $<10^{-3}$ \\ 
%       Organico  & 1 & 0.09 & 41.72 & 2 & 835 &  $<10^{-3}$ \\ 
%       Tempo & 1 & 0.92 & 4504.77 & 2 & 835 &  $<10^{-3}$  \\ 
%       Tempo$^2$& 1 & 0.35 & 227.06 & 2 & 835 &  $<10^{-3}$ \\ 
%       Residui & 836 &  &  &  &  &  \\ 
%       \hline
%     \end{tabular}
%   \end{table}
% \end{frame}

% \begin{itemize}[<+->]
%   \small    
% \item differenza, sia per valori assoluti che per varianza, tra le
%   due sessioni dovute a diverse condizioni di misura (operatori,
%   condizioni di campo)
% \item suddivisione in tre diversi range di profondit\`a:
%   \begin{itemize}
%   \item effetto anno attenuato (CO.Fzo pi\`u duro) %0-20cm
%   \item anche qui come sopra       %\item 20-40 cm
%   \item in profondit\`a la varianza si
%     attenua %40-80 cm
%   \end{itemize}
% \end{itemize}
% \end{columns}
% \end{frame}

% \begin{frame}
%   \begin{figure}[ht]
%     \includegraphics[width=0.8\textwidth, page=8]{../grafici/penetrometria/Penetrometria2015-2016.pdf}
%   \end{figure}
% \end{frame}


% \begin{frame}
%   \begin{figure}[ht]
%     \includegraphics[width=0.8\textwidth, page=9]{../grafici/penetrometria/Penetrometria2015-2016.pdf}
%   \end{figure}
% \end{frame}

% \begin{frame}
%   \begin{figure}[ht]
%     \includegraphics[width=0.8\textwidth, page=10]{../grafici/penetrometria/Penetrometria2015-2016.pdf}
%   \end{figure}
% \end{frame}

% \begin{frame}{Sintesi finale}
%   \begin{itemize}[<+->]
%   \item La densit\`a per grandi volumi mostra una maggior compattezza
%     negli appezzamenti OO. Differenza significativa, ma solo sulla
%     seconda cifra decimale. Nessun effetto delle lavorazioni
%   \item La densit\`a misurata sugli aggregati mostra l'inverso (OO
%     meno compatto). Anche qui differenze sulla seconda decimale. Le
%     lavorazioni mostrano un effetto (aratura la pi\`u compatta)
%   \item Le misure di stabilit\`a indicano un suolo generalmente poco
%     stabile (aggregati inferiori a 1 mm) e che il CO produce aggregati
%     di maggiori dimensioni (sia secchi che umidi).
%   \item La penetrometria indica un marcato effetto dell'anno (=
%     sessione di misura) attribuibile alle condizioni meteo e agli
%     operatori. L'appezzamento CO.F.zo risulta sempre pi\'u duro
%     rispetto agli altri. Analisi composizionale dubbia.
%   \item Porosimetria da analizzare
%   \end{itemize}

% \end{frame}
% \section{Porosimetria a mercurio}




%%% Local Variables:
%%% mode: latex
%%% TeX-master: t
%%% End:


\end{document}
